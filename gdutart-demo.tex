\documentclass{gdutart}

\infosetup{
  subject = 建筑给排水课程设计,
  topic   = 某八层住宅楼给排水工程设计,
  college = 土木与交通工程学院,
  major   = 建筑环境与能源应用工程,
  grade   = 2015级(1)班,
  stuid   = 3115003295,
  name    = 刘程,
  teacher = 李斌
}

\begin{document}
  \section{建筑概况}
    某八层住宅楼座落于广州市,采用钢筋混凝土框架结构,地面共八层,每层层高为3.00$m$。该楼位于城区内,地形平坦。工程地质条件良好。室外地面标高-0.30$m$。该楼的四周距墙3米处有小区给水管,管径DN150$mm$,管中心标高为-1.30$m$,能提供常年最小水压为0.45$Mpa$,供水量充足。热水由家用热水器供给,室内污废水合流排出,该楼的四周距墙6$m$处有小区污水管,管径为300$mm$,最小埋深1.5$m$。\clearpage

  \section{系统设计方案}
    \subsection{生活给水系统}
      该建筑层数为八层,层高为3$m$,因为层高未超过 3.5$m$,因此该建筑所需的水压 H 估算为:$H = 120 + (8 - 2) \times 40 = 280 Kpa$,据资料显示外网水压为450$Kpa$,远大于建筑所需的水压,所以采用直接给水的方式。

    \subsection{生活排水系统}
      根据资料可知,该建筑的室内污废水合流排出,采用单立管排水系统。其中,来自厨房和洗衣机的废水从小区上部排到小区污水网;来自厕所的合流污水通过小区下部排到化粪池,经化粪池处理后再排到小区排水管。

    \subsection{雨水排水系统}
      采用重力排水系统,屋面雨水经管道收集后排至城市雨水管网,屋面雨水经过雨水斗和雨水立管,直接排到室外雨水井,阳台雨水在连接到雨水井之前先连接到雨水口,实现间接排水。设计重现期取2年。

  \section{管道的选材与安装}
    \subsection{生活给水系统}
      给水立管采用聚乙烯管(PE),热熔接,其管径按塑料管产品规格说明书标注。室外埋地给水管采用衬里的铸铁给水管,法兰连接,管径按公称直径标注;管道穿楼板时设钢套管;每户设一水表,水表后设截止阀,一层水表井设一总表,总表后设止回阀。根据《全国民用建筑工程设计技术规范2009》(给排水篇),对于给水管道直径$\leq$200$mm$的室外给水管,距建筑物基础要求$\geq$1.0$m$。根据《室外排水设计规范》有水表井距建筑物基础$\geq$1.0$m$。

    \subsection{生活排水系统}
      排水立管采用硬聚氯乙烯塑料管(PVC-U),承插粘接,其管径按塑料管产品规格说明书标注,室外污废水横干管采用铸铁管,其管径按公称直径标注。坡度均按照《建筑给水排水设计规范》选取;

    \subsection{雨水排水系统}
      屋面雨水内排水系统由雨水斗、连接管、悬吊管、排水立管、排出管以及埋地管组成。室内雨水排水管以及出户横支管均采用硬聚氯乙烯塑料管(PVC-U),其管径按规格说明书标注,室外雨水横干管采用铸铁管,其管径按公称直径标注。其管径和坡度均按照《建筑给水排水设计规范》选取。

  \section{管道的布置与敷设}
    \subsection{生活给水系统}
      \begin{enumerate}
        \item 室外给水管道与污水管道交叉时,给水管道应敷设在上面,且接口不应重叠。
        \item 因为一层有车库,即小区周围道路为行车道,因此室外给水管道的覆土深度取0.7$m$。
        \item 根据资料可知,该楼的四周距墙3米处有小区给水管,考虑到应该尽量减少弯头,因此从小区给水管笔直布置给水入户管,并在经过楼梯间的位置布置给水立管。
        \item 从立管接入每层住户的给水管,在每层的楼梯间安装水表、阀门,入户后的管道敷设在厨侧大样图上表示。
        \item 给水横管埋地暗敷,给水立管埋墙暗敷,水表所在管段除外。
        \item 建筑物内埋地敷设的生活给水管与排水管之间的最小净距,平行埋设时不宜小于0.50$m$;交叉埋设时不应小于0.15$m$,且给水管应在排水管的上面。
      \end{enumerate}

    \subsection{生活排水系统}
      \begin{enumerate}
        \item 根据资料可知,该楼的四周距墙6米处有小区污水管。用户的厨房和洗衣机排出的废水直接从小区上部排出到小区污水管;而厕所排出的污水则通过排水管排出到位于小区下方的化粪池,经过微生物处理后再排到小区污水管。
        \item 排水管与小区排水管连接处设置检查井,检查井至建筑物距离为4$m$,检查井直径为0.7$m$。
        \item 室外排水管与给水引入管水平间距为1.5$m$以上,垂直距离0.4$m$以上;与雨水管水平间距1$m$以上,垂直距离0.1$m$以上。
        \item 排水立管上应设置检查口,检查口距离地面1$m$;因为2到8层均设有卫生器具,因此在首层和第八层设置检查口,其他楼层每隔3层设置一个。
        \item 生活排水管道的立管顶端,应设置伸顶通气管,伸顶通气管高出屋面2$m$。
        \item 室内排水立管埋墙暗敷。排水横干管埋地暗敷。
        \item 到达首层的排水立管若靠墙则直接引下埋地连接小区排水管,若未靠墙,则从首层楼板明敷到最近的墙体或柱子,然后靠墙引下埋地连接小区排水管。
        \item 设计排水管的覆土深度为1.2$m$。
      \end{enumerate}

    \subsection{雨水排水系统}
      \begin{enumerate}
        \item 雨水排水系统采用外排水系统、单斗雨水排水系统。屋面雨水系统经过立管直接与雨水井相连,而为了防止雨水回流,阳台雨水先经过室外雨水口再接入雨水检查井中。
        \item 因为雨水排水管在给水管下,且与给水管垂直敷设,因此雨水排水管与给水管的垂直间距要大于0.15$m$,所以设计覆土深度为1$m$。
      \end{enumerate} \newpage

  \section{管道水力计算}
    \subsection{生活给水系统水力计算}
      \subsubsection{室内给水系统计算}
        根据资料可知,该建筑属于\RNum{2}类普通住宅,共八层,每层各有四户。根据《广东省用水定额DB44T1461-2014》,取用水定额$q_L = 200 \makebox{升/(人$\cdot$日)}$,小时变化系数$K_h = 2.5$,设每户有3.5人,共计98人,用水时间为24$h$。各卫生器具的给水当量:洗涤盆、洗脸盆(0.75)、大便器(0.50)、淋浴器(0.50),洗衣机水嘴(1.00)。接着给最不利环路的管段标号,如下图:
        \begin{center}
          \includegraphics[scale=0.8]{affix/tubes_1.pdf}
        \end{center}
        对每段管段,按以下方法求解:
        \begin{enumerate}
          \item 计算各管段当量$N_g$;
          \item 根据以下公式计算各管段平均出流概率$U_o$:
            \begin{equation}
              {U_o} = \frac{{100{q_L}m{K_h}}}{{0.2 \cdot {N_g} \cdot T \cdot 3600}}(\% )
            \end{equation}
          \item 根据以下公式计算各管段同时出流概率$U$:
            \begin{equation}
              U = 100\frac{{1 + {\alpha _c}{{({N_g} - 1)}^{0.49}}}}{{\sqrt {{N_g}} }}(\% )
            \end{equation}
          \item 根据以下公式计算各管段设计秒流量$q_g$:
            \begin{equation}
              {q_g} = 0.2 \cdot U \cdot {N_g}
            \end{equation}
          \item 根据计算所得的各管段设计秒流量$q_g$,通过《给水塑料管水力计算表》和《生活管道水流速度》选择各管段的管径、流速和水力坡度,按公式$h = i \cdot L$计算沿程水头损失。
        \end{enumerate}
        具体计算结果如下表:
        \begin{center}
          \tiny\begin{tabularx}{\textwidth}{|Y|Y|Y|*{8}{c|}}
            \hline
            \makecell{管段} & \makecell{当量\\$N_g$} & $\alpha_c$ & \makecell{平均出\\流概率\\$U_o(\%)$} & \makecell{同时出\\流概率\\$U(\%)$} & \makecell{设计秒\\流量$q_g$\\$(L/s)$} & \makecell{管径DN\\$(mm)$} & \makecell{流速$v$\\$(m/s)$} & \makecell{水力\\坡降$i$\\$(kPa/m)$} & \makecell{管长$L$\\$(m)$} & \makecell{沿程水\\头损失\\$h_y(kPa)$} \bigstrut \\
            \hline
            1-2   & 0.50  &       &       & 100.00  & 0.10  & 15.00  & 0.50  & 0.275  & 1.04  & 0.29  \bigstrut\\
            \hline
            2-3   & 1.00  & 0.06489  & 10.13  & 100.00  & 0.20  & 15.00  & 0.99  & 0.940  & 2.62  & 2.46  \bigstrut\\
            \hline
            3-4   & 1.75  & 0.04629  & 5.79  & 78.63  & 0.28  & 20.00  & 0.79  & 0.422  & 3.47  & 1.46  \bigstrut\\
            \hline
            4-5   & 3.50  & 0.01939  & 2.89  & 55.08  & 0.39  & 25.00  & 0.61  & 0.188  & 1.21  & 0.23  \bigstrut\\
            \hline
            5-6   & 4.50  & 0.01512  & 2.25  & 48.46  & 0.44  & 25.00  & 0.61  & 0.188  & 2.83  & 0.53  \bigstrut\\
            \hline
            6-7   & 5.25  & 0.01097  & 1.93  & 44.62  & 0.47  & 25.00  & 0.76  & 0.279  & 7.11  & 1.98  \bigstrut\\
            \hline
            7-8   & 10.50  & 0.01097  & 1.93  & 31.88  & 0.67  & 32.00  & 0.69  & 0.181  & 3.00  & 0.54  \bigstrut\\
            \hline
            8-9   & 15.75  & 0.01097  & 1.93  & 26.23  & 0.83  & 32.00  & 0.79  & 0.229  & 3.00  & 0.69  \bigstrut\\
            \hline
            9-10  & 21.00  & 0.01097  & 1.93  & 22.86  & 0.96  & 32.00  & 0.98  & 0.340  & 3.00  & 1.02  \bigstrut\\
            \hline
            10-11 & 26.25  & 0.01097  & 1.93  & 20.56  & 1.08  & 32.00  & 0.98  & 0.340  & 3.00  & 1.02  \bigstrut\\
            \hline
            11-12 & 31.50  & 0.01097  & 1.93  & 18.86  & 1.19  & 32.00  & 0.98  & 0.340  & 3.00  & 1.02  \bigstrut\\
            \hline
            12-13 & 36.75  & 0.01097  & 1.93  & 17.54  & 1.29  & 40.00  & 0.90  & 0.217  & 2.80  & 0.61  \bigstrut\\
            \hline
            13-14 & 73.50  & 0.01097  & 1.93  & 12.71  & 1.87  & 40.00  & 1.20  & 0.361  & 5.05  & 1.82  \bigstrut\\
            \hline
            14-15 & 110.25  & 0.01097  & 1.93  & 10.57  & 2.33  & 50.00  & 0.95  & 0.517  & 17.75  & 9.18  \bigstrut\\
            \hline
            15-16 & 147.00  & 0.01097  & 1.93  & 9.29  & 2.73  & 50.00  & 1.14  & 0.245  & 2.25  & 0.55  \bigstrut\\
            \hline
            16-17 & 148.25  & 0.01097  & 1.93  & 9.25  & 2.74  & 50.00  & 1.14  & 0.245  & 11.97  & 2.93  \bigstrut\\
            \hline
          \end{tabularx}
        \end{center}
        通过上表计算可得引入管起点至最不利配水点的总沿程水头损失:$h = 26.34 kPa$。

      \subsubsection{压力校核计算}
        已知室外给水引入管埋深1$m$,楼高3$m$,最不利配水点与楼板高差1$m$,则由最不利配水点与引入管起点的高程差产生的静压差为:$H_1 = 1 + 3 \times 7 + 1 = 23.00m = 230.00kPa$。

        由上述计算已知引入管起点至最不利配水点的总水头损失$H_2$,局部损失按沿程损失30%计,则总水头损失$H_2 = 26.34 \times 1.3 = 34.24kPa$;

        对于通过水表产生的水头损失,可根据以下公式计算:
        \begin{equation}
          h = \frac{{100 \times {q_g}^2}}{{{q_{\max }}^2}}
        \end{equation}
        \begin{enumerate}[label=\large{\textcircled{\small{\arabic*}}}]
          \item 入户水表装在6-7管段上,该管段设计秒流量$q_g = 0.47L/s = 1.69m^3/h$,通过查表选用LXS-20C旋翼湿式水表,公称口径为20$mm$,最大流量为5$m^3/h$,常用流量为2.5$m^3h$。则用上述公式计算得水流经过水表的水头损失为11.42$kPa$;
          \item 引入管水表装在16-17管段上,该管段设计秒流量$q_g = 2.74L/s = 9.86m^3h$,通过查表选用LXS-40C旋翼湿式水表,公称直径为40$mm$,最大流量为20$m^3/h$,常用流量10$m^3/h$,则用上述公式计算得水流经过水表的水头损失为24.30$kPa$。
        \end{enumerate}
        因此设计流量通过水表时产生的水头损失$H_3 = 11.42 + 24.30 = 35.72kPa$。而对于最不利点配水附件所需最低工作压力$H_4$,取$H_4 = 100kPa$;

        综上所述,可得$H = H_1 + H_2 + H_3 + H_4 = 399.96Kpa < 450kPa$,符合当前小区管道直接供水的方案。

      \subsubsection{入户压力分布}
        首先对最不利环路的入户压力进行计算,确认是否需要装设减压阀:
        \begin{center}
          \begin{tabularx}{\textwidth}{|Y|Y|c|Y|}
            \hline
            楼层 & 静水压$(kPa)$ & 总水头损失$(kPa)$ & 进户压力$(kPa)$ \bigstrut \\
            \hline
            一层 & 24  & 34.04 & 391.96 \bigstrut \\
            \hline
            二层 & 54  & 43.92 & 352.08 \bigstrut \\
            \hline
            三层 & 84  & 52.95 & 313.05 \bigstrut \\
            \hline
          \end{tabularx}
        \end{center} 

        通过上表可知,一层入户和二层入户压力都超过了350$kPa$,因此都要设置减压阀;同时再计算三层进户压力最大处压力为328.14$kPa$,因此三层无需设置减压阀。
        \newpage

    \subsection{生活排水系统水力计算}
      \subsubsection{室内外排水系统计算}
        根据规范可知,室内排水设计秒流量计算公式为:
        \begin{equation}
          {q_p} = 0.12 \cdot \alpha  \cdot \sqrt {{N_p}}  + {q_{\max }}
        \end{equation}
        
        其中$\alpha$取1.5,$q_{\max}$是计算管段上排水量最大的一个卫生器具的排水流量。室内排水管段标注如下所示:
        \begin{center}
          \includegraphics[scale=0.8]{affix/tubes_2.pdf}
        \end{center}

        首先,根据《建筑给水排水设计规范》可确定排水接户管管径为DN160,最小设计坡度为0.005;排水干管管径为DN200,最小设计坡度为0.004。

        \paragraph{卫生器具排水管管径确定}
          根据《建筑给水排水设计规范》可得所需卫生器具的排水当量、排水流量以及排水管径:
          \begin{center}
            \begin{tabularx}{\textwidth}{|Y|c|c|c|}
              \hline
              卫生器具 & 排水当量 & 排水流量$(L/s)$ & 排水管径$(mm)$ \bigstrut\\
              \hline
              洗脸盆   & 0.75  & 0.25  & 32 \bigstrut\\
              \hline
              洗涤盆   & 1     & 0.33  & 50 \bigstrut\\
              \hline
              淋浴器   & 0.45  & 0.15  & 50 \bigstrut\\
              \hline
              大便器   & 4.5   & 1.5   & 100 \bigstrut\\
              \hline
              洗衣机 & 1.5   & 0.5   & 50 \bigstrut\\
              \hline
            \end{tabularx}
          \end{center}

        \paragraph{厨房排水管管径确定}
          厨房排水横支管,连接洗涤盆,选用管径50$mm$,通用坡度0.025。排水立管的当量总数为7.00,计算得立管最下部管段排水设计秒流量$q_g = 0.12 \times 1.5 \times \sqrt{7} + 0.33 = 0.80L/s$。流量不超过排水塑料管最大允许排水流量$0.8L/s$,故可采用伸顶通气。立管底部与排出管放大一号管径,故取管径De75$mm$。
          \begin{center}
            \begin{tabularx}{\textwidth}{|Y|*{6}{c|}}
              \hline
              管段 & 当量$N_g$ & $q_g(L/s)$ & 管径$De(mm)$ & 坡度$i$ & 管长$L(m)$ & 坡降$(m)$ \bigstrut\\
              \hline
              1-2 & 1.00 & 0.80 & 50 & 0.025 & 1.32 & 0.033 \bigstrut \\
              \hline
            \end{tabularx}
          \end{center}

        \paragraph{洗衣房排水管管径确定}
          洗衣房排水横支管,连接洗衣机,选用管径50$mm$,通用坡度0.025。排水立管的当量总数为10.50,计算得立管最下部管段排水设计秒流量$q_g = 0.12 \times 1.5 \times \sqrt{10.5} + 0.5 = 1.08L/s$。查表可得,立管管径选用De50$mm$,流量接近排水塑料管最大允许排水流量$1.0L/s$,故可采用伸顶通气。立管底部与排出管放大一号管径,故取管径De75$mm$。
          \begin{center}
            \begin{tabularx}{\textwidth}{|Y|*{6}{c|}}
              \hline
              管段 & 当量$N_g$ & $q_g(L/s)$ & 管径$De(mm)$ & 坡度$i$ & 管长$L(m)$ & 坡降$(m)$ \bigstrut\\
              \hline
              1-2 & 1.50 & 1.08 & 50 & 0.025 & 1.33 & 0.033 \bigstrut \\
              \hline
            \end{tabularx}
          \end{center}

        \paragraph{主卧室厕所排水管管径确定}
          主卧室厕所排水支管,由于连接大便器,因此选用最小管径De100$mm$,通用坡度0.025;排水立管的当量总数为39.90,计算得立管最下部管段排水设计秒流量$q_g = 5.64L/s$。查表可得,立管管径选用De100$mm$,流量小于排水塑料管最大允许排水流量$3.2L/s$,故可采用伸顶通气。立管底部与排出管放大一号管径,故取管径De110$mm$。

          \begin{center}
            \begin{tabularx}{\textwidth}{|Y|*{6}{c|}}
              \hline
              管段 & 当量$N_g$ & $q_g(L/s)$ & 管径$De(mm)$ & 坡度$i$ & 管长$L(m)$ & 坡降$(m)$ \bigstrut\\
              \hline
              1-4   & 0.75  & 0.25  & 32.00  & 0.025  & 0.79  & 0.02  \bigstrut\\
              \hline
              3-4   & 0.45  & 0.15  & 50.00  & 0.025  & 2.03  & 0.05  \bigstrut\\
              \hline
              4-5   & 5.70  & 1.90  & 100.00  & 0.025  & 0.77  & 0.02  \bigstrut\\
              \hline
            \end{tabularx}
          \end{center}

        \paragraph{餐厅厕所排水管管径确定} 
          餐厅排水横支管,由于连接大便器,因此选用最小管径De100$mm$,通用坡度0.025;排水立管的当量总数为39.90,计算得立管最下部管段排水设计秒流量$q_g = 5.64L/s$。查表可得,立管管径选用De100$mm$,流量小于排水塑料管最大允许排水流量$3.2L/s$,故可采用伸顶通气。立管底部与排出管放大一号管径,故取管径De110$mm$。

          \begin{center}
            \begin{tabularx}{\textwidth}{|Y|*{6}{c|}}
              \hline
              管段 & 当量$N_g$ & $q_g(L/s)$ & 管径$De(mm)$ & 坡度$i$ & 管长$L(m)$ & 坡降$(m)$ \bigstrut\\
              \hline
              1-4   & 0.75  & 0.25  & 32.00  & 0.025  & 2.33  & 0.06  \bigstrut\\
              \hline
              3-4   & 0.45  & 0.15  & 50.00  & 0.025  & 1.02  & 0.03  \bigstrut\\
              \hline
              4-5   & 5.70  & 1.90  & 100.00  & 0.025  & 0.36  & 0.01  \bigstrut\\
              \hline
            \end{tabularx}
          \end{center}

        \paragraph{值班室排水管管径确定}
          值班室排水横干管,2-3管段由于连接大便器,因此选用最小管径De100$mm$,通用坡度0.025;1-2管段连接洗脸盆,可知排水流量为0.25$L/s$,排水管径选32$mm$。
          
      \subsubsection{污废水检查井}
        因为污废水排水管覆土深度为1.2$m$,因此可知污废水检查井初始井底标高为:$H = -(1.2m + De + \mbox{管段坡降})$,采用内径为0.7$m$的检查井。这之后的每个检查井的井底标高即井前排水管的管底标高。 
        \begin{center}
          \small\begin{xltabular}{\textwidth}{|Y|c|c|c|c|c|c|c|}
            \hline
            编号 & 管径$(m)$ & 坡度 & 管长$(m)$ & \makecell{起点管底\\标高(m)} & \makecell{终点管底\\标高(m)} & 坡降$(m)$ & \makecell{检查井\\井底标高\\(m)} \bigstrut\\
            \hline
            W1    & 0.16  & 0.005  & 5.87  & -1.368  & -1.397  & 0.029  & -1.397  \bigstrut\\
            \hline
            W2    & 0.20  & 0.004  & 11.50  & -1.397  & -1.443  & 0.046  & -1.443  \bigstrut\\
            \hline
            W3    & 0.20  & 0.004  & 6.55  & -1.443  & -1.469  & 0.026  & -1.469  \bigstrut\\
            \hline
            W4    & 0.20  & 0.004  & 3.37  & -1.469  & -1.483  & 0.013  & -1.483  \bigstrut\\
            \hline
            W5    & 0.20  & 0.004  & 12.34  & -1.483  & -1.532  & 0.049  & -1.532  \bigstrut\\
            \hline
            W6    & 0.20  & 0.004  & 3.37  & -1.532  & -1.545  & 0.013  & -1.545  \bigstrut\\
            \hline
            W7    & 0.20  & 0.004  & 0.92  & -1.545  & -1.549  & 0.004  & -1.549  \bigstrut\\
            \hline
            W8    & 0.20  & 0.004  & 3.37  & -1.549  & -1.562  & 0.013  & -1.562  \bigstrut\\
            \hline
            W9    & 0.20  & 0.004  & 12.34  & -1.562  & -1.611  & 0.049  & -1.611  \bigstrut\\
            \hline
            W10   & 0.20  & 0.004  & 3.37  & -1.611  & -1.624  & 0.013  & -1.624  \bigstrut\\
            \hline
            W11   & 0.16  & 0.005  & 5.92  & -1.368  & -1.398  & 0.030  & -1.398  \bigstrut\\
            \hline
            W12   & 0.20  & 0.004  & 4.87  & -1.398  & -1.417  & 0.019  & -1.417  \bigstrut\\
            \hline
            W13   & 0.20  & 0.004  & 3.34  & -1.417  & -1.430  & 0.013  & -1.430  \bigstrut\\
            \hline
            W14   & 0.20  & 0.004  & 4.88  & -1.430  & -1.450  & 0.020  & -1.450  \bigstrut\\
            \hline
            W15   & 0.20  & 0.004  & 6.90  & -1.450  & -1.478  & 0.028  & -1.478  \bigstrut\\
            \hline
            W16   & 0.20  & 0.004  & 4.87  & -1.478  & -1.497  & 0.019  & -1.497  \bigstrut\\
            \hline
            W17   & 0.20  & 0.004  & 3.34  & -1.497  & -1.510  & 0.013  & -1.510  \bigstrut\\
            \hline
            W18   & 0.20  & 0.004  & 4.88  & -1.510  & -1.530  & 0.020  & -1.530  \bigstrut\\
            \hline
          \end{xltabular}
        \end{center} \newpage

    \subsection{雨水排水系统水力计算}
      \subsubsection{暴雨强度计算}
        广州市暴雨强度公式为:
        \begin{equation}
          q = \frac{{3618.427(1 + 0.438\lg P)}}{{{{(t + 11.259)}^{0.750}}}}(L/s \cdot {10^4}{m^2})
        \end{equation}
        其中暴雨重现期P取2年,降雨历时t取5$min$,则通过该公式计算可得$q = 505.8L/s \cdot {10^4}{m^2}$。

      \subsubsection{屋面雨水设计流量}
        屋面汇水总面积约为400$m^2$,总共有8个雨水斗,因此每个汇水面积为50$m^2$,屋面径流系数$\Phi = 0.9$。根据以下公式计算雨水设计流量:
        \begin{equation}
          {q_y} = \frac{{{q_j}\psi {F_w}}}{{10000}}(L/s)
        \end{equation}
        因此根据上述公式解得$q_y = 2.28L/s$,即单根立管雨水量为$2.28L/s$。

        根据上述计算值,查屋面雨水斗的最大泄流量得重力流排水时,采用87式雨水斗,口径为75$mm$,最大排水能力为8$L/s$,大于设计排水量1.76$L/s$以及2.93$L/s$,符合要求。根据《给水排水设计手册》第二册,选用管径75$mm$的立管,其最大排水量为10$L/s$,符合要求。悬吊管管径与立管管径相同。立管底部与排出管放大一号管径,故取管径100$mm$。

      \subsubsection{阳台雨水设计流量}
        根据《给水排水设计手册》第二册,阳台的汇水面积由阳台面积加上侧墙面积的一半,卧室阳台计算结果为8.96$m^2$,客厅阳台计算结果为13.5$m^2$。

        暴雨强度和雨水量计算方法同上,代入公式计算得出卧室阳台单根立管雨水量为0.41$L/s$;客厅阳台单根立管雨水量为0.61$L/s$。

        根据《给水排水设计手册》第二册,卧室阳台每根立管雨水量为0.41$L/s$,选用管径75$mm$的立管,其最大排水量为10$L/s$;客厅阳台每根立管雨水量为0.61$L/s$,选用管径75$mm$的立管,其最大排水量为10$L/s$。立管底部与排出管放大一号管径,故取管径110$mm$。

      \subsubsection{室外雨水管管径选择}
        根据《建筑给水排水设计规范》,小区建筑物周围的雨水接户管最小管径为200$mm$,横管最小设计坡度为0.003;小区道路下的干管最小管径为300$mm$,最小设计坡度为0.0015。

      \subsubsection{雨水检查井}
        因为雨水排水管覆土深度为1$m$,所以雨水检查井初始井底标高为:$H=-(1.0m + De + \mbox{管段坡降})$,采用内径为0.7$m$的检查井。这之后的每个检查井的井底标高即井前雨水管的管底标高。 
        \begin{center}
          \small\begin{xltabular}{\textwidth}{|Y|c|c|c|c|c|c|c|}
            \hline
            编号 & 管径$(m)$ & 坡度 & 管长$(m)$ & \makecell{起点管底\\标高(m)} & \makecell{终点管底\\标高(m)} & 坡降$(m)$ & \makecell{检查井\\井底标高\\(m)} \bigstrut\\
            \hline
            Y1    & 0.20  & 0.0030  & 2.11  & -1.368  & -1.374  & 0.006  & -1.374  \bigstrut\\
            \hline
            Y2    & 0.30  & 0.0015  & 5.87  & -1.374  & -1.383  & 0.009  & -1.383  \bigstrut\\
            \hline
            Y3    & 0.30  & 0.0015  & 14.13  & -1.383  & -1.404  & 0.021  & -1.404  \bigstrut\\
            \hline
            Y4    & 0.30  & 0.0015  & 5.87  & -1.404  & -1.413  & 0.009  & -1.413  \bigstrut\\
            \hline
            Y5    & 0.20  & 0.0030  & 4.60  & -1.413  & -1.427  & 0.014  & -1.427  \bigstrut\\
            \hline
            Y6    & 0.30  & 0.0015  & 1.81  & -1.427  & -1.430  & 0.003  & -1.430  \bigstrut\\
            \hline
            Y7    & 0.30  & 0.0015  & 5.20  & -1.430  & -1.438  & 0.008  & -1.438  \bigstrut\\
            \hline
            Y8    & 0.30  & 0.0015  & 12.99  & -1.438  & -1.457  & 0.019  & -1.457  \bigstrut\\
            \hline
            Y9    & 0.30  & 0.0015  & 3.73  & -1.457  & -1.463  & 0.006  & -1.463  \bigstrut\\
            \hline
            Y10   & 0.30  & 0.0015  & 3.28  & -1.463  & -1.468  & 0.005  & -1.468  \bigstrut\\
            \hline
          \end{xltabular}
        \end{center} \newpage

  \section*{参考文献}
    \addcontentsline{toc}{section}{参考文献}
    \begin{enumerate}[label={[\arabic*]}]
      \item GB/T50106-2010,建筑给水排水制图标准[S]. 北京:中国建筑工业出版社,2010。
      \item GB/T50001-2010,房屋建筑制图统一标准[S]. 北京:中国计划出版社,2011。
      \item 住房和城乡建设部. 建筑工程设计文件编制深度规定(2016年版)[Z]. 北京:中国计划出版社,2016。
      \item GB50015-2003,建筑给水排水设计规范(2009年版)[S]. 北京:中国计划出版社,2010。
      \item 王增长主编. 建筑给水排水工程(第七版)[M]. 北京:中国建筑工业出版社,2017。
      \item 给水排水国家标准图集 (S1$\sim$S6). 北京:中国建筑工业出版社。
      \item 全国民用建筑工程设计技术措施-给水排水(第二版)2008[M]. 北京:中国建筑工业出版社,2008。
      \item 张智主编. 给排水科学与工程专业毕业设计指南(第二版)[M]. 北京:中国水利水电出版社,2008。
      \item 其他设计资料、产品样本、设计手册、期刊论文等。
    \end{enumerate}
\end{document}